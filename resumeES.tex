%!TEX TS-program = xelatex
%!TEX encoding = UTF-8 Unicode
% Plantilla LaTeX de Awesome CV
%
% Licencia de la plantilla:
% CC BY-SA 4.0 (https://creativecommons.org/licenses/by-sa/4.0/)
%

%%%%%%%%%%%%%%%%%%%%%%%%%%%%%%%%%%%%%%
%     Configuración
%%%%%%%%%%%%%%%%%%%%%%%%%%%%%%%%%%%%%%
%%% Tema: Awesome-CV
\documentclass[]{awesome-cv}
\usepackage{textcomp}
%%% Sobrescribe la ubicación de los directorios de fuentes (por defecto: 'fonts/')
\fontdir[fonts/]

%%% Configura la ubicación del directorio de secciones
\newcommand*{\sectiondir}{resume/}

%%% Sobrescribe el color
% Colores Awesome: awesome-emerald, awesome-skyblue, awesome-red, awesome-pink, awesome-orange
%                 awesome-nephritis, awesome-concrete, awesome-darknight
%% Color para resaltar
% Define tu propio color personalizado si no te gustan los colores de Awesome
\colorlet{awesome}{awesome-red}
%\definecolor{awesome}{HTML}{CA63A8}
%% Colores para el texto
%\definecolor{darktext}{HTML}{414141}
%\definecolor{text}{HTML}{414141}
%\definecolor{graytext}{HTML}{414141}
%\definecolor{lighttext}{HTML}{414141}

%%% Sobrescribe el separador para la información social en la cabecera (por defecto: ' | ')
%\headersocialsep[\quad\textbar\quad]

\begin{document}

%%%%%%%%%%%%%%%%%%%%%%%%%%%%%%%%%%%%%%
%     Perfil
%%%%%%%%%%%%%%%%%%%%%%%%%%%%%%%%%%%%%%
\begin{center}
	\headerfirstnamestyle{Facundo} \headerlastnamestyle{Giacconi} \\
	\vspace{2mm}
	{\faEnvelope\ fgiacconi.dev@gmail.com}  |  {\faMobile\ (+34) 679122423}  |  {\faMapMarker\ Girona, España} 
	{\faLinkedin\ \href{https://www.linkedin.com/in/facundo-giacconi-fernandez-a77989a3}{Facundo}}
\end{center}

\cvsection{Habilidades}
\begin{cventries}
	\cventry
	{}
	{\def\arraystretch{1.15}{\begin{tabular}{ l l }
		Lenguajes:  & {\skill{ Java, JavaScript, TypeScript, C\#, SQL, HTML, CSS, Ruby on Rails, Jsp}} \\
		Frameworks:  & {\skill{ Spring, Spring Boot, AngularJs, Angular (6-13), React, Nextjs}} \\
		Bases de datos:  & {\skill{ MySQL, PostgreSQL, Oracle, SQLite, MongoDb}} \\
		Tecnologías / Herramientas: \hspace{0.05cm} & {\skill{ Docker, Jenkins, RabbitMQ, Kafka, SonarQube, Maven, Gradle, REST, Java JPA, Hibernate, npm}}\\
        \hspace{0.1cm} & {\skill{ Yarn, Git, Gitlab, Subversion.}}\\
		Prácticas:  & {\skill{ Agile, Scrum, Desarrollo Dirigido por Pruebas, CI/CD, Revisiones de Código.}} \\
		\end{tabular}}}
	{}
	{}
	{}
\end{cventries}
\vspace{-7mm}

%%%%%%%%%%%%%%%%%%%%%%%%%%%%%%%%%%%%%%
%     Experiencia
%%%%%%%%%%%%%%%%%%%%%%%%%%%%%%%%%%%%%%
\cvsection{Experiencia}
\begin{cventries}
	\cventry
	{Desarrollador Full Stack e I+D}
	{Azimut Electronics}
	{Gandía, España}
	{2022/04 – Actualidad}
	{\begin{cvitems}
		\vspace{0.5mm}
		\item {Trabajando de forma remota en Gandía, desarrollando y mejorando programas multimedia y de entretenimiento digital diseñados para mantener entretenidos a los clientes durante el transporte público a larga distancia.}
		\item {Proyecto basado en el frontend AngularJs y el backend Spring Boot con conectividad a MongoDB y comunicaciones REST.}
        \item {Además, a cargo de otras aplicaciones basadas en el framework NextJs con React y JavaScript como desarrollador principal con backend NodeJs con sequelize, graphQl y Mysql.}
        \item {Además, colaboración con creación de otras aplicaciones de herramientas de apoyo basadas en Python (flask, tkinter, pymongo).}
		\end{cvitems}}

	\cventry
	{Desarrollador Full Stack}
	{Sopra Steria}
	{Madrid, España}
	{2021/08 – 2022/04}
	{\begin{cvitems}
		\vspace{0.5mm}
		\item {Trabajando de forma remota en Austria, desarrollador Full Stack en una compañía de seguros.}
		\item {Desarrollo y mejora de un proyecto de gestión de productos de seguros, con frontend Angular 6 y backend Spring Boot, comunicación entre microservicios con Apache Kafka y Zookeeper y comunicaciones REST.}
		\end{cvitems}}

	\cventry
	{Desarrollador Full Stack}
	{DXC Technology}
	{Girona, España}
	{2016/11 – 2021/08}
	{\begin{cvitems}
		\vspace{0.5mm}
		\item {Desarrollo en un proyecto de licitación pública gubernamental basado en el frontend Angular y el backend Spring Boot, utilizando búsquedas Elasticsearch y persistencia de datos en Postgres.}
  \item {Desarrollo en un proyecto público gubernamental basado en Jsp con frontend HTML, Javascript y CSS, y backend Spring Boot, utilizando el motor Elasticsearch y persistencia de datos en SQL Server.}
  \item {Desarrollador en la transición que tuvo la aplicación de frontend con tecnología Html, Jpa y Jquery a frontend Angular 6}
		\end{cvitems}}

	\cventry
	{Desarrollador Full Stack}
	{Coditramuntana}
	{Girona, España}
	{2016/02 – 2016/11}
	{\begin{cvitems}
		\vspace{0.5mm}
		\item {Desarrollo de diferentes tipos de proyectos CRM y SPA basados en Ruby on Rails con MySQL como base de datos de persistencia.}
		\end{cvitems}}


	\cventry
	{Desarrollador Full Stack}
	{Terramar Tours}
	{Lloret De Mar, España}
	{2013/06 – 2016/02}
	{\begin{cvitems}
		\vspace{0.5mm}
		\item {Desarrollo y mantenimiento/mejora en proyecto diseñado para crear diferentes horarios de viaje automáticamente (schedules) a partir de las necesidades de los usuarios. Proyecto MVC basado en el backend Java Spring con persistencia de datos en SQL y frontend en JSP, CSS y comunicaciones REST.}
  \item {Desarrollador y mantenimiento/mejora en el sector de relaciones de API de un proyecto basado en Spring y Java en el backend, y JSP y CSS en el frontend, con una base de datos SQL y la biblioteca JOOQ.}
		\end{cvitems}}
\end{cventries}

%%%%%%%%%%%%%%%%%%%%%%%%%%%%%%%%%%%%%%
%     Proyectos
%%%%%%%%%%%%%%%%%%%%%%%%%%%%%%%%%%%%%%
\vspace{35mm}
\cvsection{Proyectos}
\begin{cventries}	
	\vspace{1mm}
	\cventry
	{}
	{Pizzapp}
	{Java 17, Spring Boot 3.2.0, Angular 17, NodeJs 5, Websockets, MongoDB, Redis, Docker \vspace{-5mm}}
	{}
	{\begin{sectionnormaltext}
		\item {Aplicación web desarrollada como un sistema de pedidos de pizza, con un frontend Angular 17 y un backend administrado por microservicios con Spring Boot, NodeJs, Kafka, Zookeeper, MongoDB, Redis y Websockets. 
		\newline \faLink\ \href{https://github.com/flautarian/pizzapp}{Github repository}}
	\end{sectionnormaltext}}

	\vspace{-3mm}
	\cventry
	{}
	{Owncloud \vspace{-5mm}}
	{Java 8, Spring Boot 2.4.7, AngularJs 1.5, MongoDB 4.4.2, Gitlab \vspace{-5mm}}
	{}
	{\begin{cvsectionnormaltext}
		\item {Aplicación web empresarial e interna desarrollada como CRM para que el personal gestione, controle y supervise grupos de autobuses equipados con tecnología de Azimut y su contenido multimedia. 
		\newline \faLink\ \href{https://www.azimutelectronics.com}{Página de inicio empresarial}}
	\end{cvsectionnormaltext}}

	\vspace{-3mm}
	\cventry
	{}
	{Herramienta de simulación de Allianz \vspace{-5mm}}
	{Java 8, Spring Boot 2.4, Apache Kafka, Angular 6, Subversion \vspace{-5mm}}
	{}
	{\begin{cvsectionnormaltext}
		\item{Aplicación web desarrollada como herramienta de asistente para crear y calcular provisionalmente las pólizas para la empresa Allianz.
      \newline \faLink\ \href{https://www.allianzdirect.es/seguro-de-coche/calcular-precio/}{Simulador de pólizas}}
	\end{cvsectionnormaltext}}
	
	\vspace{-3mm}
	\cventry
	{}
	{Plataforma de servicios de contratación pública \vspace{-5mm}}
	{Java 8, Spring Boot 2, Elasticsearch, Angular 6, Github, PostgreSQL \vspace{-5mm}}
	{}
	{\begin{cvsectionnormaltext}
		\item{Desarrollador desde el principio de la nueva plataforma en Angular 6 al mismo tiempo que se desarrollaba la plataforma actual.
		\newline \faLink\ \href{https://contractaciopublica.cat/ca/inici}{Plataforma de servicios de contratación pública}}
	\end{cvsectionnormaltext}}

    \cventry
	{}
	{API multiproposito basada en Python con Microsoft Azure \vspace{-5mm}}
	{Python 3, Azure functions, Mongodb 6, Flask, Github \vspace{-5mm}}
	{}
	{\begin{cvsectionnormaltext}
		\item{Aplicación multi propósito creada para tener una API backend online y privada para aplicaciones personales gracias a Azure Functions, con un stack basado en Python, con acceso por identificación Auth y sistema de tokens temporales, para almacenar información de manera autónoma de diferentes aplicaciones que lo necesiten en una base de datos Mongodb 6 y tengan acceso a dicha API.
		\newline}
	\end{cvsectionnormaltext}}

    \cventry
	{}
	{TrelloApp \vspace{-5mm}}
	{React 18, Typescript, Sonner, Beaufitul-DnD, Github, Vercel, reactI18Next \vspace{-5mm}}
	{}
	{\begin{cvsectionnormaltext}
		\item{Aplicación Trello like creada para afianzar y refrescar conocimientos sobre React, dispone de conectividad y de sistema de Login por sesión token para acceder a las listas creadas por el usuario.
		\newline \faLink\ \href{https://trello-app-giacconidev.vercel.app/user/login}{TrelloApp}}
	\end{cvsectionnormaltext}}
 
	\vspace{-5mm}
\end{cventries}

%%%%%%%%%%%%%%%%%%%%%%%%%%%%%%%%%%%%%%
%     Educación
%%%%%%%%%%%%%%%%%%%%%%%%%%%%%%%%%%%%%%
\vspace{8mm}
\cvsection{Educación}
\begin{cventries}
	\vspace{2mm}
	\cventry
	{}
	{Instituto Montilivi \vspace{-5mm}}
	{Girona, España \vspace{-5mm}}
	{}
	{\begin{cvsectionnormaltext} 
		\item{Educación superior en Administración y Desarrollo de Redes del Sistema (HNC)}
	\end{cvsectionnormaltext}}

 \cventry
	{}
	{Universidad Abierta de Cataluña \vspace{-5mm}}
	{Girona, España \vspace{-5mm}}
	{}
	{\begin{cvsectionnormaltext} 
		\item{Grado en ingeniería informática}
	\end{cvsectionnormaltext}}
\end{cventries}


%%%%%%%%%%%%%%%%%%%%%%%%%%%%%%%%%%%%%%
%     Cursos
%%%%%%%%%%%%%%%%%%%%%%%%%%%%%%%%%%%%%%
\vspace{25mm}
\cvsection{Cursos}
\begin{cventries}
	\vspace{2mm}
    \cventry
	{}
	{A hundred Python projects \vspace{-4mm}}
	{En línea \vspace{-5mm}}
	{}
	{\begin{cvsectionnormaltext} 
		\item{Curso con más de 70 horas de contenido, con los principios del lenguaje de programación Python, y con la generación de 100 proyectos con dicho lenguaje, desde aplicaciones nativas copn Tkinter hasta webapps y backends con Flask.
		\newline \vspace{2mm} \faLink\ \href{https://www.udemy.com/certificate/UC-2d01021e-af92-434c-a4a3-7fc7b33976c9/}{Certificado de finalización}}
	\end{cvsectionnormaltext}}
    {}
    
	\cventry
	{}
	{Curso de NextJs \vspace{-4mm}}
	{En línea \vspace{-5mm}}
	{}
	{\begin{cvsectionnormaltext} 
		\item{Curso con más de 40 horas de contenido sobre todo lo necesario para desarrollar una aplicación web stateless con React y NextJs.
		\newline \vspace{2mm} \faLink\ \href{https://www.udemy.com/certificate/UC-1ede8757-aa0d-406f-813a-8eaa400532c5/}{Certificado de finalización}}
	\end{cvsectionnormaltext}}
    {}

	\cventry
	{}
	{Curso de React desde lo básico hasta aplicaciones avanzadas \vspace{-4mm}}
	{En línea \vspace{-5mm}}
	{}
	{\begin{cvsectionnormaltext} 
		\item{Curso de React desde lo básico hasta aplicaciones avanzadas (Hooks, MERN) con más de 50 horas de contenido.
		\newline \vspace{2mm} \faLink\ \href{https://www.udemy.com/certificate/UC-fcb467ab-c089-419c-9b6a-afe97f894d14/}{Certificado de finalización}}
	\end{cvsectionnormaltext}}
    {}

	\cventry
	{}
	{Universidad de Spring - Framework Spring y Spring Boot \vspace{-4mm}}
	{En línea \vspace{-5mm}}
	{}
	{\begin{cvsectionnormaltext} 
		\item{Curso de Spring Boot realizado para refrescar los conocimientos sobre Spring Boot y su tecnología (+70h)
		\newline \vspace{2mm} \faLink\ \href{https://www.udemy.com/certificate/UC-ee0ce349-6915-479b-b038-5253aba9d0d8/}{Certificado de finalización}}
	\end{cvsectionnormaltext}}
    {}

	\cventry
	{}
	{Guía completa del desarrollo en MongoDb 2022 \vspace{-4mm}}
	{En línea \vspace{-5mm}}
	{}
	{\begin{cvsectionnormaltext} 
		\item{Curso genérico con conocimientos básicos sobre Js, Jquery, Angular 9 y NodeJs con diferentes ejemplos y buenas prácticas.
		\newline \vspace{2mm} \faLink\ \href{https://www.udemy.com/certificate/UC-b990e98a-9b76-44b1-8e3e-43b68213725c/}{Certificado de finalización}}
	\end{cvsectionnormaltext}}
    {}
    
	\cventry
	{}
	{Curso de refresco de patrones de diseño Java \vspace{-4mm}}
	{En línea \vspace{-5mm}}
	{}
	{\begin{cvsectionnormaltext} 
		\item{Curso de refresco de patrones mas usados en el lenguaje Java, junto a ejemplos y casos practicos de uso.
		\newline \vspace{2mm} \faLink\ \href{https://www.udemy.com/certificate/UC-982e3338-ea7d-4434-bf27-af8e38b77bd5/}{Certificado de finalización}}
	\end{cvsectionnormaltext}}
    {}

    
	\cventry
	{}
	{Curso de Js, Jquery, Angular 9 y NodeJs \vspace{-4mm}}
	{En línea \vspace{-5mm}}
	{}
	{\begin{cvsectionnormaltext} 
		\item{Curso genérico con conocimientos básicos sobre Js, Jquery, Angular 9 y NodeJs con diferentes ejemplos y buenas prácticas.
		\newline \vspace{2mm} \faLink\ \href{https://www.udemy.com/certificate/UC-012febdf-7657-4e65-8708-ee4e18d5a8e1/}{Certificado de finalización}}
	\end{cvsectionnormaltext}}
    {}
\end{cventries}

%%%%%%%%%%%%%%%%%%%%%%%%%%%%%%%%%%%%%%
%     Otra información
%%%%%%%%%%%%%%%%%%%%%%%%%%%%%%%%%%%%%%
\vspace{-5mm}
\cvsection{Información de interés}
\begin{cventries}
	\cventry
	{}
	{}
	{}
	{}
	{\begin{cvsectionnormaltext} 
		\item{Inglés: B2 
        \newline \vspace{1mm} Español: Nativo 
        \newline \vspace{1mm} Catalán: Nativo}
	\end{cvsectionnormaltext}}
	\cventry
	{Licencia de conducir: B}
	{}
	{}
	{}
    {}
\end{cventries}
\end{document}
